\documentclass{article}

\usepackage{afterpage}
\usepackage{fontspec}
\usepackage{geometry}
\usepackage{hyperref}
\usepackage{lscape}
\usepackage{pgfgantt}
\usepackage{siunitx}
\usepackage{titling}

\newcommand*{\subtitle}[1]{\gdef\thesubtitle{#1}}

\title{CRAPS Kernel}
\subtitle{Specifications}
\author{
       Maxime Arthaud
  \and Korantin Auguste
  \and Martin Carton
  \and Étienne Lebrun
  \and Pierre-Louis Michel
}

\begin{document}
  \begin{titlepage}
  \begin{center}
    \includegraphics[height=1cm]{LogoEnseeiht}\\\vspace{1cm}
    \hrule\vspace{0.5cm}
    \textsc{\Large\thesubtitle}
    \\\vspace{0.5cm}

    \textbf{\huge\thetitle}
    \\\vspace{0.4cm}
    \hrule\vspace{2cm}

    {\large
      Maxime~\textsc{Arthaud}      \\
      Korantin~\textsc{Auguste}    \\
      Martin~\textsc{Carton}       \\
      Étienne~\textsc{Lebrun}
    }

    \vfill
    {\large January -- March 2015}
  \end{center}
\end{titlepage}

  \newpage
    \section{Basic Functionalities}
      The first version consists of the most basic OS functionalities.

      \subsection{Improvement of the Processor}
        \begin{enumerate}
          \item The processor shall have access to at least \SI{32}{kB} of RAM.
          \item The RAM shall be accessed in less than 10 cycles.
          \item The processor shall handle several interruptions sources, with
            a priority system. An interruption handler can only be interrupted
            by an interruption of a strictly higher priority.
          \item Several registers shall be added.
        \end{enumerate}

      \subsection{Compiler}
        To write the OS in reasonable time, we need to be able to compile a
        higher level language to CRAPS assembly. For that purpose, we chose to
        reuse the compiler we wrote last year.
        \begin{enumerate}
          \item The compiler input shall be a subset of \textit{C89} (see
            appendix~\ref{app:C89} for details).
          \item The compiler shall generate valid CRAPS assembly.
        \end{enumerate}

      \subsection{Development of the OS}
        \begin{enumerate}
          \item The OS shall have a scheduler. There are no real-time
          constraint (but it shall switch regularly, at least 10 times per
          second). It shall support at least 10 hard-coded tasks.
          \item The OS shall provide a basic library available to any program:
            \begin{itemize}
              \item a \verb+malloc+ to allocate dynamic memory;
              \item functions to write on the LEDs and seven-segment display,
                and read the switches;
              \item division and modulus operations.
             \end{itemize}
        \end{enumerate}

    \section{RS-232 Version}

      \subsection{Improvement of the Processor}
        \begin{enumerate}
          \item The processor shall be connected to a serial port, to allow
            communication with other devices (like a computer). It can use the
            \textit{RS-232} port on the FPGA.
        \end{enumerate}

      \subsection{Compiler}
        \begin{enumerate}
          \item Enlarge the subset of \textit{C89}
          \item The assembly generated shall be more optimized.
        \end{enumerate}

      \subsection{Development of the OS}
        \begin{enumerate}
          \item The OS shall have a task communicating with the user through
            the \textit{RS-232}, in the form of a very simple shell.
          \item Extension to the OS library: an API to read/write to the serial
            port.
        \end{enumerate}
      \subsection{Tools}
        \begin{enumerate}
          \item The monitor shall be available in Linux.
          \item The monitor shall display all the RAM.
          \item The CRAPS assembler tool shall be available without a FPGA connected to the computer (it is the case with the tool we were given).
          \item Adapt the tools to make them work on Linux (they are working only on Windows for now).
        \end{enumerate}

    \section{Version with persistent storage}
      If we have enough time, it would be interesting to access the flash memory
      present on the board. This implies the development of a SHDL/VHDL module
      and the OS code to use it. With this improvements, we could:
      \begin{itemize}
        \item write and read memory from the shell;
        \item implement a basic file system.
      \end{itemize}
      Ideally, we could write a tutorial for the students, explaining the steps
      to realise their own OS.

  \newpage

  \begin{appendix}
    \section{Description of the C subset}\label{app:C89}
      The C subset available at the beginning of the project includes
      (non-exhaustive list):
      \begin{itemize}
        \item \textit{while} loop
        \item \textit{if} loop (no \textit{else if})
        \item 32-bits integer arithmetic (without division, modulo and only
          16-bits multiplication)
        \item functions
        \item an \text{x86} back-end
      \end{itemize}

      \noindent The \textit{V1} shall add:
      \begin{itemize}
        \item a \text{CRAPS} back-end
        \item global variables
        \item exported functions
        \item \textit{for} loop
        \item missing operators (address operator, bitwise operators, bit shift)
      \end{itemize}

      \noindent The following elements are planned for the \textit{V2}:
      \begin{itemize}
        \item dynamic memory
        \item arrays
        \item structures
        \item the \verb+&+ (address) operator
      \end{itemize}
      We will add more functionalities if they prove to be needed to write the
      OS. 
  \end{appendix}

\end{document}
