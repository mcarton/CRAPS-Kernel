\documentclass{article}

\usepackage{afterpage}
\usepackage{fontspec}
\usepackage{geometry}
\usepackage{hyperref}
\usepackage{lscape}
\usepackage{paralist}
\usepackage{siunitx}
\usepackage{titling}

\newcommand*{\subtitle}[1]{\gdef\thesubtitle{#1}}

\title{CRAPS Kernel}
\subtitle{Test Plan}
\author{
       Maxime Arthaud
  \and Korantin Auguste
  \and Martin Carton
  \and Étienne Lebrun
}

\begin{document}
  \begin{titlepage}
  \begin{center}
    \includegraphics[height=1cm]{LogoEnseeiht}\\\vspace{1cm}
    \hrule\vspace{0.5cm}
    \textsc{\Large\thesubtitle}
    \\\vspace{0.5cm}

    \textbf{\huge\thetitle}
    \\\vspace{0.4cm}
    \hrule\vspace{2cm}

    {\large
      Maxime~\textsc{Arthaud}      \\
      Korantin~\textsc{Auguste}    \\
      Martin~\textsc{Carton}       \\
      Étienne~\textsc{Lebrun}
    }

    \vfill
    {\large January -- March 2015}
  \end{center}
\end{titlepage}

  \tableofcontents
  \newpage

  \section{Introduction and Objectives}
    This document presents the testing strategies we used to test the tools
    produced for the CRAPS Kernel project. Three main tools have to be tested:
    \begin{inparaenum}[\itshape a\upshape)]
      \item the processor,
      \item the compiler,
      \item the kernel itself.
    \end{inparaenum}

    The tests for these tools are independent. But the kernel depends on both
    the processor and the compiler.

  \section{Objectives}
    \subsection{Processor Tests}
      The goals of these tests is to prove that every processor instruction
      works as intended. We will only test the features we have added to the
      processor given by Jean-Christophe Buisson, since he has already tested
      this version.

      Stress tests for the RS-232 will be needed to check that no data is lost.

    \subsection{Compiler Tests}
      The goals of these tests is to prove the code generated by the compiler is
      correct. We will have to tests every \textit{C} structure the compiler
      supports.

    \subsection{Kernel Tests}
      The goals of these tests is to prove the kernel is correct.
\end{document}
