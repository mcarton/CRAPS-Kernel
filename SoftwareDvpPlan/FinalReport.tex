\documentclass[openany]{book}

\usepackage{afterpage}
\usepackage{fontspec}
\usepackage{geometry}
\usepackage{hyperref}
\usepackage{lscape}
\usepackage{siunitx}
\usepackage{titling}

\newcommand*{\subtitle}[1]{\gdef\thesubtitle{#1}}

\title{CRAPS Kernel}
\subtitle{Final Report}
\author{
       Maxime Arthaud
  \and Korantin Auguste
  \and Martin Carton
  \and Étienne Lebrun
  \and Pierre-Louis Michel
}

\begin{document}
  \begin{titlepage}
  \begin{center}
    \includegraphics[height=1cm]{LogoEnseeiht}\\\vspace{1cm}
    \hrule\vspace{0.5cm}
    \textsc{\Large\thesubtitle}
    \\\vspace{0.5cm}

    \textbf{\huge\thetitle}
    \\\vspace{0.4cm}
    \hrule\vspace{2cm}

    {\large
      Maxime~\textsc{Arthaud}      \\
      Korantin~\textsc{Auguste}    \\
      Martin~\textsc{Carton}       \\
      Étienne~\textsc{Lebrun}
    }

    \vfill
    {\large January -- March 2015}
  \end{center}
\end{titlepage}


  \chapter*{Acknowledgments}
    We would like to thank Jean-Christophe Buisson for giving us all the tools
    related to CRAPS that he has developed. We won a lot of time thanks to this.

    \paragraph{}
    We also would like to thank Bernard Desmyter and Benoît Lemarchand for
    having tested our compiler and showed us some bugs we did not see ourselves.

    \paragraph{}
    Finally, we would like to thank Xavier Mechin, our industrial supervisor,
    for guiding us throughout this two-month project.

  \tableofcontents

  \chapter{Introduction}
    \section{Context}
      This project, suggested by Daniel Hagimont, is based on the CRAPS
      processor developed by Jean-Christophe Buisson and used in the first-year
      CPU architecture courses at ENSEEIHT. The goal is to develop an operating
      system that would run on top of that processor.

      The reasons for that project are that before, it was only possible to do a
      little of assembly directly on the processor to see it work, but nothing
      more.  After our project, it should be possible for students to really see
      the layer that goes on top of the CPU in modern computers: the operating
      system. So that the students can really make the link between the
      processor they just built and the computer and underlying operating
      systems they use everyday.

    \section{Work required}
    \section{Available resources}

  \chapter{Project management}

  \chapter{Specification phase}

  \chapter{Implementation phase}

  \chapter{Validation phase}
    \section{Tests}
    \section{Limitations}

  \chapter{Conclusion}

  \addcontentsline{toc}{chapter}{Bibliography}
  \begin{thebibliography}{10}
  \end{thebibliography}
\end{document}
