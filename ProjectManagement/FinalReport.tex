\documentclass[openany]{book}

\usepackage{afterpage}
\usepackage{fontspec}
\usepackage{geometry}
\usepackage{hyperref}
\usepackage{lscape}
\usepackage{siunitx}
\usepackage{titling}

\newcommand*{\subtitle}[1]{\gdef\thesubtitle{#1}}

\title{CRAPS Kernel}
\subtitle{Final Report}
\author{
       Maxime Arthaud
  \and Korantin Auguste
  \and Martin Carton
  \and Étienne Lebrun
  \and Pierre-Louis Michel
}

\begin{document}
  \begin{titlepage}
  \begin{center}
    \includegraphics[height=1cm]{LogoEnseeiht}\\\vspace{1cm}
    \hrule\vspace{0.5cm}
    \textsc{\Large\thesubtitle}
    \\\vspace{0.5cm}

    \textbf{\huge\thetitle}
    \\\vspace{0.4cm}
    \hrule\vspace{2cm}

    {\large
      Maxime~\textsc{Arthaud}      \\
      Korantin~\textsc{Auguste}    \\
      Martin~\textsc{Carton}       \\
      Étienne~\textsc{Lebrun}
    }

    \vfill
    {\large January -- March 2015}
  \end{center}
\end{titlepage}


  \chapter*{Acknowledgments}
    We would like to thank Jean-Christophe Buisson for giving us all the tools
    related to CRAPS that he has developed. We won a lot of time thanks to this.

    \paragraph{}
    We also would like to thank Bernard Desmyter and Benoit Lemarchand for
    having tested our compiler and showed us some bugs we did not see ourselves
    and Mickaël Carl for helping us with VHDL's best practices.

    \paragraph{}
    Finally, we would like to thank Xavier Mechin, our industrial supervisor,
    for guiding us throughout this two-month project.

  \tableofcontents

  \chapter{Introduction}
    \section{Context}
      This project, suggested by Daniel Hagimont, is based on the CRAPS
      processor developed by Jean-Christophe Buisson and used in the first-year
      CPU architecture courses at ENSEEIHT. The goal is to develop an operating
      system that would run on top of that processor.

      The reasons for that project are that before, it was only possible to do a
      little of assembly directly on the processor to see it work, but nothing
      more.  After our project, it should be possible for students to really see
      the layer that goes on top of the CPU in modern computers: the operating
      system. So that the students can really make the link between the
      processor they just built and the computer and underlying operating
      systems they use everyday.

    \section{Work required}
    \section{Available resources}
      Jean-Christophe Buisson initially lent us 2 boards. As this was not enough
      for the four of us to work efficiently, he lent us 4 more boards.
      He also gave us the sources of the tools he developed for his students so
      that we could adapt them for our needs.

      Our client got us a room in ENSEEIHT to work.

      \paragraph{}
      We also set up a \textit{git}\footnote{\textit{git} is distributed Version
      Control System (VCS).} repository to put all the sources and document we
      have produced during this project. This allowed us to efficiently work
      simultaneously on the same files and keep a history of all changes.
      We also set up a mailing list and a Jabber room\footnote{\textit{Jabber}
      is an instant messaging protocol.} to communicate with each other.

      We did all the work on our own personal computers.

  \chapter{Project management}
    The team is composed of four members:
    \begin{itemize}
      \item Korantin Auguste as \textit{developer}
      \item Maxime Arthaud as \textit{tester}
      \item Martin Carton as \textit{project leader}
      \item Étienne Lebrun as \textit{quality manager}
    \end{itemize}

    We were initially 5 but one of us had to leave the team for medical reasons
    at the beginning of the project.

  \chapter{Specification phase}
    We planned a meeting with the client the first day of the project. He
    explained us what he wanted us to do and that the project was very flexible.
    His goal is to be able to make lab sessions for first-year students of
    ENSEEIHT. Thus, as long as he can show students how an operating system
    works the project will be a success.

  \chapter{Implementation phase}

  \chapter{Validation phase}
    \section{Tests}
    \section{Limitations}

  \chapter{Conclusion}

  \addcontentsline{toc}{chapter}{Bibliography}
  \begin{thebibliography}{9}
    \bibitem{buisson}
      Jean-Christophe Buisson,
      \emph{Introduction à l'architecture des ordinateurs},
      Chapitre VII.\ CRAPS~: guide du programmeur.

      \mbox{\url{http://diabeto.enseeiht.fr/download/archi/cours/book2_4.4.pdf}}

    \bibitem{nexys2}
      \emph{Digilent Nexys2 Board, Reference Manual},
      Revision: July 11, 2011.

      \mbox{\url{http://www.digilentinc.com/data/products/nexys2/nexys2_rm.pdf}}
  \end{thebibliography}

  \addcontentsline{toc}{chapter}{Annexes}
  \chapter*{Annexes}
    \newgeometry{top=1cm,bottom=1cm}
    \thispagestyle{empty}
    \begin{figure}
      \centering
      \includegraphics[
        angle=90,
        width=\linewidth,
        height=\textheight
      ]{Gantt.pdf}
      \label{fig:gantt}
    \end{figure}
    \clearpage
    \restoregeometry
\end{document}
