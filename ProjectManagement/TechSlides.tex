\documentclass{beamer}

% {{{ beamer stuffs
\setbeamertemplate{footline}[frame number]
\beamertemplatenavigationsymbolsempty%

\AtBeginSection[]{%
  \begin{frame}<beamer>{Plan}
    \tableofcontents[currentsection]
  \end{frame}
}
\AtBeginSubsection[]{%
  \begin{frame}<beamer>{Plan}
    \tableofcontents[currentsubsection]
  \end{frame}
}
% }}}

\usepackage{fontspec}
\usepackage{hyperref}
\usepackage[french]{babel}

\title{CRAPS Kernel}
\subtitle{Présentation technique}
\author{
       Maxime Arthaud
  \and Korantin Auguste
  \and Martin Carton
  \and Étienne Lebrun
}
\titlegraphic{\includegraphics[width=0.5\textwidth]{fig/LogoEnseeiht.png}}
\date{5 avril 2014}

\begin{document}
  \begin{frame}
    \titlepage%
  \end{frame}

  \section{Compilateur}
    \begin{frame}[fragile]{Compilateur}
      \begin{itemize}
        \item Basé sur notre projet compilateur de 2A (utilisant Chicken, une
          version modifiée de EGG). Nouveau backend CRAPS.
        \item Syntaxe proche du C (voir doc).
        \item Support des includes via un petit wrapper avec préprocesseur:
          \verb+compile_craps+.
      \end{itemize}
\end{frame}

    \begin{frame}[fragile]{Gère}
      \begin{itemize}
        \item Boucles \verb+while+, \verb+for+, conditions;
        \item Structures, tableaux;
        \item \verb+sizeof()+;
        \item Déréférencement, opérateur \verb+&+;
        \item Variables globales;
        \item Typage fort (et cast).
      \end{itemize}
\end{frame}

    \begin{frame}[fragile]{Ne gère pas}
      \begin{itemize}
        \item Pas de déréférencement des champs des structures,
        \item Pas d'opérateurs \verb|++|, \verb|--|,
        \item Pas d'opérateurs \verb|+=|, \verb|-=|, \verb|*=|, \dots;
        \item Pas de division/modulo (à cause du processeur),
        \item Pas de switch/case.
      \end{itemize}

      \pause
      → Voir \verb+Compiler/doc/language.pdf+.
\end{frame}

    \begin{frame}[fragile]{Optimisations}
      \begin{itemize}
        \item Suppression des fonctions non utilisées (cf.\ mot clé
          \verb+export+);
        \item Simplifications des expressions arithmétiques;
        \item Calcul d'adresse à la compilation;
        \item Utilisation de \verb+%r0+ et \verb+%r20+ autant que possible;
        \item Utilisation de \verb+sethi+, \verb+setq+, \verb+set+;
        \item Simplification des \verb+while(true)+, \verb+if+ sans else, etc…
      \end{itemize}
\end{frame}

    \begin{frame}[fragile]{Gestion des registres}
      \begin{itemize}
        \item Les 19 registres banalisés sont utilisés
        \item Support du mot clé \verb+register+
      \end{itemize}
\end{frame}

  \begin{frame}[fragile]{Position-independent code}
    \begin{itemize}
      \item Registre spécial \verb+%r24+;
      \item Contient la position des variables et chaine de caractères
        statiques;
      \item Au lancement d'un processus, un petit bout de code assembleur
        l'initialise.
    \end{itemize}
\end{frame}

  \section{Système d'exploitation}
    \begin{frame}[fragile]{Vue d'ensemble}
      \begin{itemize}
        \item Pas d'espace user/noyau;
        \item Pas de séparation mémoire;
        \item Tout se base sur des interruptions\dots
        \item \dots dont les appels systèmes
      \end{itemize}
\end{frame}

    \begin{frame}[fragile]{Ordonnanceur}
      \begin{itemize}
        \item Basé sur les interruptions PWM matérielles;
        \item Table des processus (pointeur vers la pile);
        \item Round-robin.
      \end{itemize}
\end{frame}

    \begin{frame}[fragile]{Gestion mémoire}
      \begin{itemize}
        \item \verb+malloc+/\verb+free+/\verb+realloc+;
        \item implémentation très simple (taille puissance de deux, pas de
          merge/split des blocs, mais réutilisation);
        \item conservation du PID du processus qui alloue la mémoire, et routine
          pour libérer toute la mémoire allouée d'un processus.
      \end{itemize}
\end{frame}

    \begin{frame}[fragile]{Démarrage}
      Fonction \verb+main+, à montrer.
\end{frame}

  \section{Modifications du processeur, VHDL}
    \begin{frame}[fragile]{Documentation}
      Tout a été documenté dans le dossier \verb+Processor/doc+.
      \begin{itemize}
        \item Interruptions;
        \item Test and set;
        \item Registres;
        \item RS232;
        \item RAM.
      \end{itemize}
\end{frame}

    \begin{frame}[fragile]{Gestion des interruptions}
      \begin{itemize}
        \item Ajout d'un module \verb+interruptions+ dans la micromachine;
        \item Encodeur de priorité 4 bits;
        \item Avant chaque fetch-decode, le sequenceur vérifie s'il y a une
          interruption de plus haute priorité en attente;
        \item Ajout d'une interruption software (modification de \verb+crapsc+);
        \item Table d'interruptions à l'adresse \verb+1+: adresses des handlers
          d'interruption.
      \end{itemize}
\end{frame}

    \begin{frame}[fragile]{Test and set}
      \begin{itemize}
        \item finalement pas utilisée;
        \item voir \verb+Processor/doc/registers.md+;
        \item le format de l'instruction \verb+reti+ a été changé.
      \end{itemize}
\end{frame}

    \begin{frame}{Registres}
      \begin{itemize}
        \item 0: 0
        \item 1--19: registres banalisés
        \item 20: 1
        \item 21, 22: registres temporaires pour le processeur
        \item 23: non utilisé
        \item 24: PIC, position des globales (nouveau)
        \item 25: PSR, pour interruptions (nouveau)
        \item 26: breakpoint
        \item 27: FP: current's frame base pointer
        \item 28: ret
        \item 29: SP, top of stack
        \item 30: PC, next instruction
        \item 31: IR, current instruction
      \end{itemize}
    \end{frame}

    \begin{frame}[fragile]{RS232}
      \begin{itemize}
        \item Lecture en \verb+0xD0000000+, écriture en \verb+0xD0000001+;
        \item Interruption 5: quand on reçoit un bit;
        \item Interruption 6: quand on a fini d'écrire un bit (ne marche pas);
          \pause
        \item du coup écriture puis attente.
      \end{itemize}
\end{frame}

    \begin{frame}[fragile]{Mémoire}
      \begin{itemize}
        \item Utilisation de la \verb+BRAM+, CRAPS dispose maintenant de 48Ko;
        \item RAM 16Mo mappée aux adresses \verb+0xE0******+;
        \item La RAM est plus lente que la BRAM.
      \end{itemize}
\end{frame}

  \section{Pratique}
    \begin{frame}[fragile]{Synthèse}
      \begin{itemize}
        \item Sur Windows uniquement (programme pas porté);
        \item utilise le script \verb+shdl2vhdl+;
        \item → cocher "synthesize" puis glisser/déposer le ficher;
        \item attendre 15mins.
      \end{itemize}
\end{frame}

    \begin{frame}{Flasher une carte}
      \begin{itemize}
        \item Sur Windows uniquement (pas testé sur Linux);
        \item Via Digilent Adept: deux choix:
          \begin{itemize}
            \item Sur le FPGA: rapide, volatile;
            \item Sur la ROM: long, persistant.
          \end{itemize}
      \end{itemize}
    \end{frame}

    \begin{frame}{Uploader du code avec interface graphique}
      Montrer.
    \end{frame}

    \begin{frame}[fragile]{Uploader du code avec crapsdb}
      \begin{verbatim}
% ./compile-craps path/to/kernel.moc
% ./crapsc path/to/kernel.asm
% ./crapsdb path/to/kernel.obj
Loading file path/to/kernel.obj ... done.
[… machine's state …]
> break 0x42
> run
Breakpoint
[… machine's state …]
> print %r30
%r30 = 291
> exit
      \end{verbatim}
\end{frame}

\end{document}
