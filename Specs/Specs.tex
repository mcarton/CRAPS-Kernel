\documentclass{article}

\usepackage{afterpage}
\usepackage{fontspec}
\usepackage{geometry}
\usepackage{hyperref}
\usepackage{lscape}
\usepackage{pgfgantt}
\usepackage{siunitx}
\usepackage{titling}

\title{CRAPS Kernel Specifications}
\author{
       Maxime Arthaud
  \and Korantin Auguste
  \and Martin Carton
  \and Étienne Lebrun
  \and Pierre-Louis Michel
}

\begin{document}
  \begin{titlepage}
  \begin{center}
    \includegraphics[height=1cm]{LogoEnseeiht}\\\vspace{1cm}
    \hrule\vspace{0.5cm}
    \textsc{\Large\thesubtitle}
    \\\vspace{0.5cm}

    \textbf{\huge\thetitle}
    \\\vspace{0.4cm}
    \hrule\vspace{2cm}

    {\large
      Maxime~\textsc{Arthaud}      \\
      Korantin~\textsc{Auguste}    \\
      Martin~\textsc{Carton}       \\
      Étienne~\textsc{Lebrun}
    }

    \vfill
    {\large January -- March 2015}
  \end{center}
\end{titlepage}

  \newpage

  \section{Specifications}

    \begin{enumerate}
    \item Improvement of the processor
        \begin{enumerate}
        \item The processor should support at least 32 kB of RAM. The RAM should be accessed in a fast way, but the loss of a few (less than 10) cycles is acceptable.
        \item The processor should support the I/O needed to read/write from/to permanent storage (SD card or persistent internal memory)
        \item The processor should support a serial port that can be opened for communicating with another device (like a computer). It will use the RS232 port on the FPGA.
        \end{enumerate}
    \item Compiler
        \begin{enumerate}
        \item The compiler should support a langage similar as C89, allowing us to easily develop the operating system.
        \item The compiler should generate valid CRAPS assembly.
        \end{enumerate}
    \item Development of the OS
        \begin{enumerate}
        \item The OS should have a scheduler. There are no real-time constraint (but it should switch regularly, at least 10 times per second). It should support at least 10 hard-coded tasks.
        \item The OS should have a task making a led blink and showing that it is working and that the scheduler runs.
        \item The OS should have another task communicating with the user, in the form of a VERY simple shell. At the end, this mini-shell will allow writing and reading to permanent storage.
        \item The OS should have a few functions available to any program : at least a "malloc" to allocate dynamic memory, and an API to read/write to the serial port and to permanent storage.
        \end{enumerate}
    \end{enumerate}

\end{document}
